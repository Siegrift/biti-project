\chapter{Bundlers and Trusted Types}

\label{chapter:bundlers} % chapter id for ref command

% TODO: this will need to be rewritten for master thesis
We have briefly explained the problem with bundlers when TT are enforced in development mode. The
purpose of the project was to take a look at example application using many modern bundlers and see
how they integrate with Trusted Types. This work was based off my previous experience when doing the
integration for Webpack \cite{webpack_tt_integration}.

When bundlers are not TT compatible it often means \textbf{hard blocker} for the application
developer to start using TT. This is because the application in dev environment might not even load
\emph{(this is the case when using webpack)} or some features such as hot reloading and error
widgets might not work.

The integration for webpack is already in progress and will hopefully get merged soon. I've instead looked at the following:

\begin{itemize}
  \item  ESBuild \cite{esbuild_web} - ESBuild is an extremely fast JavaScript bundler by factor
        10-100x compared to popular bundler alternatives. The main reason for such performance is
        because it's written in Go and heavily uses parallelism \cite{esbuild_fast}.
  \item  Snowpack \cite{snowpack_web} - Snowpack is an alternative to heavier, more complex bundlers
        like Webpack or Parcel in your development workflow. Snowpack leverages JavaScript's native
        module system called \emph{(ESM - ES modules)} which makes it scale much better as the
        project grows on size.
  \item  Vite \cite{vite_web} - Vite is very similar to Snowpack and uses the same idea - \emph{use
          the native ES modules}. The main difference is that Vite is opinionated about
        \emph{production builds} and uses \emph{Rollup} under the hood. This provides simpler
        developer experience and unlocks features which are not natively supported by Snowpack.
  \item  WMR \cite{wmr_web} - WMR is a tiny development tool composed in a single 2mb file with no
        dependencies. It provides a nice integration with Preact, but can be used independently as
        well.
\end{itemize}

As a bonus I also tried creating application using the current state of the art React project
starters \emph{(which use Webpack as bundler internally)}:

\begin{itemize}
  \item  NextJS \cite{nextjs_web} - NextJS is a React framework, which provides many features such
        as extended server side rendering, hosting, deployments, etc...
  \item  CRA (Create React App) \cite{cra_web} - CRA is the official recommended way by React to
        bootstrap new application which want to use React. It uses very well configured Webpack
        under the hood and keeps in sync with latest React versions.
\end{itemize}

% Add vertical space, because the next paragraph will be an intro for the sections followed.
\bigskip

In order to demonstrate the bundlers we need an example application. I've come across a great
article which compares the bundlers I want to integrate TT with. The article referenced also a
github project with an example application that was written using each of the mentioned bundlers. We
are not going to compare the bundlers among themselves, but rather focus on TT integration. However,
you can check the blog post at
\url{https://css-tricks.com/comparing-the-new-generation-of-build-tools/}.

\section{ESBuild}

TODO:

\section{Snowpack}

TODO:

\section{Vite}

TODO:

\section{WMR}

TODO:

\section{NextJS}

TODO:

\section{CRA}

TODO:
