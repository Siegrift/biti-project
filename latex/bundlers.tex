\chapter{Bundlers and Trusted Types}

\label{chapter:bundlers} % chapter id for ref command

% TODO: this will need to be rewritten for master thesis
We have briefly explained the problem with bundlers when TT are enforced in development mode. The
purpose of the project was to take a look at example application using many modern bundlers and see
how they integrate with Trusted Types. This work was based off my previous experience when doing the
integration for Webpack \cite{webpack_tt_integration}.

When bundlers are not TT compatible it often means \textbf{hard blocker} for the application
developer to start using TT. This is because the application in dev environment might not even load
\emph{(this is the case when using webpack)} or some features such as hot reloading and error
widgets might not work.

The integration for webpack is already in progress and will hopefully get merged soon. I've instead looked at the following:

\begin{itemize}
  \item  ESBuild \cite{esbuild_web} - ESBuild is an extremely fast JavaScript bundler by factor
        10-100x compared to popular bundler alternatives. The main reason for such performance is
        because it's written in Go and heavily uses parallelism \cite{esbuild_fast}.
  \item  Snowpack \cite{snowpack_web} - Snowpack is an alternative to heavier, more complex bundlers
        like Webpack or Parcel in your development workflow. Snowpack leverages JavaScript's native
        module system called \emph{(ESM - ES modules)} which makes it scale much better as the
        project grows on size.
  \item  Vite \cite{vite_web} - Vite is very similar to Snowpack and uses the same idea - \emph{use
          the native ES modules}. The main difference is that Vite is opinionated about
        \emph{production builds} and uses \emph{Rollup} under the hood. This provides simpler
        developer experience and unlocks features which are not natively supported by Snowpack.
  \item  WMR \cite{wmr_web} - WMR is a tiny development tool composed in a single 2mb file with no
        dependencies. It provides a nice integration with Preact, but can be used independently as
        well.
\end{itemize}

As a bonus I also tried creating application using the current state of the art React project
starters \emph{(which use Webpack as bundler internally)}:

\begin{itemize}
  \item  NextJS \cite{nextjs_web} - NextJS is a React framework, which provides many features such
        as extended server side rendering, hosting, deployments, etc...
  \item  CRA (Create React App) \cite{cra_web} - CRA is the official recommended way by React to
        bootstrap new application which want to use React. It uses very well configured Webpack
        under the hood and keeps in sync with latest React versions.
\end{itemize}

% Add vertical space, because the next paragraph will be an intro for the sections followed.
\bigskip

In order to demonstrate the bundlers we need an example application. I've come across a great
article which compares the bundlers I want to integrate TT with. The article referenced also a
github project with an example application that was written using each of the mentioned bundlers. We
are not going to compare the bundlers among themselves, but rather focus on TT integration. However,
you can check the blog post at
\url{https://css-tricks.com/comparing-the-new-generation-of-build-tools/}.

Unfortunately, the example application didn't cause any XSS on it's own because React avoids using
sinks and instead creates DOM elements dynamically. Instead, I've created a sample React component
with example attack vectors - one with and the other without trusted types policy.

All of the examples can be run locally, because you can mock CSP policy using HTML meta header which
is included in every entry point HTML file. You can find all of the projects \emph{(as well as the
  sources for this document)} in \url{https://github.com/Siegrift/biti-project}.

\section{ESBuild}

ESBuild was the first project examined, however there wasn't much to test regarding TT integration.
ESBuild only transpiles and concatenates the JS code which is then included in HTML file and server
by sample server. It doesn't provide neither code reloading nor any special error reporting.

This demonstrates that bundling itself is often not a problem, because any advanced operations such
as \emph{minification or obfuscation} should preserve the code correctness - for example, if a TT
policy is enclosed in a module \emph{(and not exported)} than it must be enclosed in the bundled
code as well.

However minification impact is a valid concern for bundlers, this seems to not be an issue as the TT
code seems to minify well \cite{tt_webpack_integration_minification}.

\section{Snowpack}

Snowpack was the second bundler examined. As mentioned it uses ES modules and avoids module
concatenation for development. As mentioned though, module concatenation isn't the root cause when
doing TT integration.

The pleasant finding was that hot reloading was working out of the box with TT integration. This is
because when a source module \emph{(JS file)} changes, it is downloaded by the browser and picked up
by the running web page.

\emph{(For example, when webpack is trying to reload the code it first has to create a bundle or
  chunk, and then it uses dynamically created script to download it from dev server. However
  \textbf{script.src} is a sink and TT disallows such assignment.)}

There was only a single TT violation caused by Snowpack error overlay widget, which is using
\emph{Element.innerHTML} internally \cite{snowpack_violation}.

\section{Vite}

Vite is very similar to Snowpack and uses ES modules same as Snowpack. It also doesn't provide an
error overlay so there is no violation for this case.

However, I found out that reloading CSS triggers TT violation, because it seems to use dynamic style
elements under the hood \cite{vite_violation}.

\section{WMR}

WMR has basically same results as ESBuild because it only bundles the code.

\section{NextJS}

Since all of the example projects targeted React and a specific \emph{modern} bundler, I've decided
to revisit the integrations with current state of the art project boilerplates/frameworks.

NextJS is a very popular choice these days, because it comes preconfigured with webpack, React,
routing, serverless and more. It is also integrated with a great CLI which allows developers to
deploy the application in just a single command \emph{(and for free as of now)} \cite{nextjs_web}.

Unfortunately, nextJS is complex and there are many places which cause TT violations:

\begin{itemize}
  \item  HTML violations - For hot reload container which is inject to page using \emph{innerHTML}
        property.
  \item  Script violations - For webpack hot reload. It downloads chunks, which then execute the
        updated code using \emph{eval}.
  \item  Script URL violations - Also for webpack hot reload.
\end{itemize}

Unfortunately, as of now, there is no ideal best practice how to differentiate these benign usages
of sinks apart from unsafe usages in developer code \cite{tt_source_file_violation_issue}.

\section{CRA}

Another very popular choice to bootstrap the project using CRA - Create react app. This is also the
official recommended way to start a React single page application.

I've also wanted to test how TT play together with browser extensions. I've created a basic crypto
app which integrates with Metamask browser extension, which is used to connect to Ethereum
blockchain. There were three types of TT violations:

\begin{itemize}
  \item  HTML violations - Caused by react error overlay component. Unfortunately, you can't
        suppress this error even with default policy, because the sink assignment happens in another
        JS realm. You can understand more in \cite{cross_document_vectors}.
  \item  Script violations - Metamask browser extension is using a Function constructor. I am not
        sure what for, because the app seems to work even when this code is disallowed. Also, due to
        a bug in chromium a workaround is needed to suppress this violation
        \cite{tt_fn_constructor}.
  \item  Script URL violations - Again, needed for webpack hot reload.
\end{itemize}

The only significant problem was the HTML violation in overlay component. This error can only
\emph{(and should)} be fixed by the library.
