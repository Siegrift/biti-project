\chapter{Results}

\label{chapter:results} % chapter id for ref command

% TODO: this will need to be rewritten for master thesis
I've compared multiple popular bundlers and project starters. Similar types of TT violations
repeated among the bundlers.

\begin{itemize}
  \item  Error overlay widget - Snowpack, CRA and NextJS use an error widget to show compile and
        application error and stacktrace to improve developer experience. Under the hood this uses
        \emph{innerHMTL} property. Developers need to use default policy with \emph{createHTML}
        function.
  \item  Script violations - Needed for next JS hot reloading which uses webpack under the hood,
        which uses \emph{eval} to evaluate the updated chunk of code. It was also needed for
        Metamask widget \emph{(not sure why though)}. Developers need to use default policy with
        \emph{createScript} function.
  \item  Script URL violations - Again, needed for webpack hot reload. The chunks to be reloaded are
        downloaded using and dynamically created script and by assigning to \emph{script.src}
        property. Developers need to use default policy with \emph{createScriptURL} function.
\end{itemize}

Also it can be seen that the bundlers which use ES modules under the hood integrate better with TT.
Snowpack doesn't need any policy for code reload and Vite only needs a policy to reload CSS.

\section{Further work and TT improvements}

I feel the biggest problem with TT is the adoption difficulty. Developers are facing difficult
issues in third party code \emph{(which is often minified)} and default policy doesn't provide
enough context to reliably allow only \emph{Trusted values}
\cite{tt_source_file_violation_issue_comment}.

Another aspect of work is to implement TT integrations inside the bundlers and project starters. The
future of TT seems bright as React, Angular, VS Code and many other project start to use and enforce
TT in their applications. Maybe in some time TT will be default standard in web development.

You can check out all the sources and run the projects locally in
\url{https://github.com/Siegrift/biti-project}.
