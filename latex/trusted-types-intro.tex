\chapter{Introduction to Trusted Types}

\label{chapter:intro-tt} % chapter id for ref command

In this chapter we are going to introduce the concept of Trusted Types (abbr. TT). The threat model
under which TT operate is DOM XSS \emph{(subset of client side XSS)}. This vulnerability happens
when untrusted, user controlled input reaches a \emph{sink} which is function like \emph{eval()} or
\emph{innerHTML} property of an HTML element.

The idea is to take all of the sinks and make them secure by default. For example, in DOM you can
assign any string value to \emph{innerHTML} property of an element and the browser will parse the
string using HTML parser and render the parsed markup. The problem is that HTML \emph{script} tags
can parse and execute arbitrary JS code. But that's not all, you can execute code with HTML
attributes and CSS too. Instead, we make these sinks accept only \emph{\textbf{special branded
    objects}} and throw an error on any other value.

This is a breaking change behaviour in DOM spec and is unfortunately not acceptable to change the
spec in this way. Instead, users who want to use TT have to explicitely opt-in to enforce this API.
As of now, TT are supported in Chromium based browser but it is likely that the support will
increase. You can check out the supported browsers in \cite{tt_compatibility}.

Developers are asked to avoid the sinks if possible to reduce the attack surface, however, sometimes
there is no way to avoid the sink. For such cases there should be a way for developers to create
these \emph{special branded objects} and make them explicitely \textbf{trusted} by the DOM APIs and
sinks - hence the name \emph{Trusted Types}. Of course, if creating TT would be as easy as wrapping
the value in a function call, developers would quickly misuse this behaviour and just wrap the
unsafe values in that function call instead of making sure the value can be really trusted. To
prevent this, there is a security mechanism guarding how these values can be created.

Refer to the recommended explainer article \cite{trusted_types_into} for more info or the formal
spec \cite{tt_spec}.

\section{Enforcing Trusted Types}

As mentioned, TT is backward incompatible change and must be enforced explicitely by the developers.
You can enable it by adding the following response header to the documents which should operate
under TT enforcement.

\bigskip
% https://tex.stackexchange.com/questions/75324/lstinputlisting-for-a-normal-text-file
\begin{lstlisting}[language={}, caption={Enforcing Trusted Types}]
  Content-Security-Policy: require-trusted-types-for 'script';
\end{lstlisting}

\section{Creating Trusted Types}

Most of the times, you can avoid using the sinks alltogether and use safe DOM APIs instead
\emph{(e.g. create DOM node dynamically instead of innerHTML assignment)}. However there are cases
when using sinks can't be avoided, or the values are already trusted, sanitized or encoded. You can
create trusted type from such value which will be accepted by the DOM sink.

If your browser supports TT, you will have access to \emph{trustedTypes} global object on
\emph{window} instance. You can use \emph{trustedTypes.createPolicy} function to create a policy,
which acts like a "factory" for creating the trusted values. For example, citing
\cite{trusted_types_into}:

\bigskip
\begin{lstlisting}[language=ES6, caption={Creating Trusted Types policy}]
  if (window.trustedTypes && trustedTypes.createPolicy) { // Feature testing
  const escapeHTMLPolicy = trustedTypes.createPolicy('myEscapePolicy', {
    createHTML: string => string.replace(/\</g, '&lt;')
  });
}
\end{lstlisting}

and then you would use the policy simply by \emph{(again citing \cite{trusted_types_into})}:

\bigskip
\begin{lstlisting}[language=ES6, caption={Using the policy to create Trusted value}]
  const escaped = escapeHTMLPolicy.createHTML('<img src=x onerror=alert(1)>');
  console.log(escaped instanceof TrustedHTML);  // true
  el.innerHTML = escaped;  // '&lt;img src=x onerror=alert(1)>'
\end{lstlisting}

However, as you noticed the policies are named. This is because you can restrict which policies
might be created by your application. All you need to do is provide additional response header with
a whitelist of allowed policy names, separated by comma.

\bigskip
% https://tex.stackexchange.com/questions/75324/lstinputlisting-for-a-normal-text-file
\begin{lstlisting}[language={}, caption={Restricting policy names}]
  Content-Security-Policy: require-trusted-types-for 'script'; trusted-types myEscapePolicy
\end{lstlisting}

\section{Report only mode}

Before switching to enforcing mode, which will cause an error when untrusted value is assigned to a
sink, you can switch to report only mode, which will preserve the original DOM API behaviour
\emph{(no error will be thrown and standard sink behaviour will be used)}. However, there will be a
warning in the console and CSP violation will be triggered, which can be sent to configured URL for
triage \cite{tt_intro_csp_violation}.

You will also have to use slightly different CSP response header \cite{trusted_types_into}:

\bigskip
% https://tex.stackexchange.com/questions/75324/lstinputlisting-for-a-normal-text-file
\begin{lstlisting}[language={}, caption={Report only CSP}]
  Content-Security-Policy-Report-Only: require-trusted-types-for 'script'; report-uri //my-csp-endpoint.example
\end{lstlisting}

\section{Default policy}

Creating your own policies and making sure \emph{trusted} values reach the sinks is not always
possible. Application code is largely composed of many dependencies and third party code, which the
developer can't easily modify. For such cases, there is special \emph{\textbf{default policy}} which
will be called when untrusted value reaches the sink. It is the last place where the developer
\emph{(specifically, the handler function the developer implemented)} can \emph{"bless"} the value
such that it will be accepted by the DOM API. \cite{trusted_types_into}

\bigskip
\begin{lstlisting}[language=ES6, caption={Creating default policy}]
  if (window.trustedTypes && trustedTypes.createPolicy) { // Feature testing
    trustedTypes.createPolicy('default', {
      createHTML: (string, sink) => DOMPurify.sanitize(string, {RETURN_TRUSTED_TYPE: true})
    });
  }
\end{lstlisting}

Note, that \emph{default policy} is still a named policy and must be listed in the allowlist
specified by \emph{trusted-types} response header \emph{(of course, this only applies if the
  application restricts the policy names)}.

\section{Trusted Types integration}

TT are being developed and used internally at Google, where they integrate it into their core
products. However, many open source projects see the benefit of TT and started doing necessary steps
needed for the integration. The amount of work needed to integrate TT to application varies a lot.
There are many factors which might contribute to this complexity:

\begin{itemize}
  \item  Choice of a bundler - This is more a developer nuisance than production issue. Main purpose for
        bundlers is to transpile the code using latest \emph{(and not widely supported)} features to
        standard version of JavaScript that all browsers understand. Nowdays, bundlers also take
        care of serving the app in the development phase and providing features such as code
        reloading \emph{(which refreshes the web page once developer code is saved)} or showing a popup
        with error message and stack trace. These development features often cause violations when
        TT are enforced.
  \item DOM framework - Modern web apps usually use some kind of framework \emph{(examples include:
          React, Angular, Polymer, Vue, Svelte...)}. These frameworks often provide fine security
        measures, but some frameworks may be more \emph{"TT friendly"} then others. For example,
        making React or Polymer TT compliant is much easier than integration with Angular
        \cite{react_tt_integration} \cite{polymer_tt_integration} \cite{angular_tt_integration}.
  \item Other third party dependencies - Many third party dependencies might use sinks under the
        hood and fixing such violations might be hard.
  \item Application code - These are the violations in the source code of the developer, these are
        usually the easiest to fix. You either avoid using the sink or wrap the value in a policy.
\end{itemize}
